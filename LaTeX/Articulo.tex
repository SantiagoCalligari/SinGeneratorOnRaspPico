\documentclass[]{article}
\usepackage{enumitem,amssymb,tikz,siunitx, pgfplots, hyperref,minted}
\usetikzlibrary{calc,matrix, arrows.meta, positioning, decorations,decorations.markings, math}
\usemintedstyle{dracula}
\usepackage[spanish]{babel} % To obtain English text with the blindtext package
\usepackage{blindtext}
\hypersetup{
	colorlinks=true,
	linkcolor=blue,
	filecolor=magenta,      
	urlcolor=blue,
	pdfpagemode=FullScreen,
}
%opening
\title{Generador De Ondas Sobre Raspberry Pi Pico}
\author{Santiago Calligari, Gerónimo Nestares}
\begin{document}
\maketitle

\section*{Preámbulo}
En este artículo vamos a contar el proceso sobre la generación de ondas sinusoidales en una placa de desarrollo, vamos a exponer todos los obstáculos que nos topamos, las soluciones que les dimos y las ideas tanto matemáticas como electrónicas que utilizamos para poder hacer esta idea una realidad que pueda ser replicada.\\
Nuestra placa de desarrollo cuenta con algunos pines \textbf{G.P.I.O.} (General Purpouse Input/Output) que son utilizados para propósitos generales de entrada/salida. Nosotros vamos a utilizar uno de estos pines para llegar a nuestra onda sinusoidal, pero nos encontramos con un problema ¡No existen las ondas analógicas en el mundo digital! Nosotros contamos solamente con una función "prendido" que nos entrega $5V$ en nuestro pin y una función "apagado" que nos entrega $0V$. Aquí entra nuestro primer concepto a explicar, la modulación por ancho de pulsos.
\section*{Modulación Por Ancho de Pulsos}
\begin{tikzpicture}
	\draw (-1.5,-1) -- (8.5,-1) -- (8.5,3.1) -- (-1.5,3.1) -- (-1.5,-1);
	\begin{axis}[
		legend style={nodes = {scale=0.5, transform shape}, at={(0.088,-0.17)}},
		legend entries={$Onda Digital$\\$Voltaje Efectivo$\\},
		ymajorgrids=true,
		width=10cm,
		height=4cm,
		x axis line style={-stealth},
		y axis line style={-stealth},
		title={Onda Cuadrada},
		ymax = 5.5,xmax=10.5,
		ymin = 0, xmin = 0,
		axis lines*=center,
		xtick distance=1,
		ytick={0,2.5,5},
		xlabel={Tiempo$\rightarrow$},
		ylabel={Voltaje},
		xlabel near ticks,
		ylabel near ticks]
		\addplot+[blue, thick, no marks ,const plot]
		coordinates{(0,5)(0.5,5)(0.5,0)(1,0)(1,5)(1.5,5)(1.5,0)(2,0)(2,5)(2.5,5)(2.5,0)(3,0)(3,5)(3.5,5)(3.5,0)(4,0)(4,5)(4.5,5)(4.5,0)(5,0)(5,5)(5.5,5)(5.5,0)(6,0)(6,5)(6.5,5)(6.5,0)(7,0) (7,5)(7.5,5)(7.5,0)(8,0)(8,5)(8.5,5)(8.5,0)(9,0)(9,5)(9.5,5)(9.5,0)(10,0)(10,5)(10.5,5)};
		\addplot+[red, very thick, no marks ,const plot]
		coordinates{(0,2.5) (11,2.5)};
\end{axis}
\end{tikzpicture}\\
Esto es una onda cuadrada, el único tipo de ondas que puede generar un microcontrolador. Desde aquí tenemos que llegar a una onda que sea un poco más parecida al seno que buscamos. Bien, esta onda particular está en 5V la mitad del tiempo y el resto se encuentra apagada ¿por qué no le ponemos un nombre al tiempo que está en $5V$? ¡Ya sé! Lo vamos a llamar Duty Cycle (Ciclo de trabajo). Lo vamos a definir como $D_{c}$ y que sea igual al porcentaje de tiempo que está en $5V$, en el caso de nuestra gráfica sería un $D_{c}$ = $50\%$.\\\\
Vamos a hacer un pequeño paréntesis para hablar de procesadores, a muy groso modo estos son un tipo de calculadoras extremadamente básicas y exorbitantemente rápidas. Dentro de sí hay un pequeño cristal que en nuestro caso oscila a $133 MHz$, que equivale a $133$ Millones de oscilaciones por segundo. Cada ciclo de reloj nuestro procesador puede tanto sumar, comparar o cargar. En el caso que nosotros encendamos nuestro pin cuando el numero de oscilaciones es par y lo apaguemos cuando sea impar, estaríamos generando una onda cuadrada de $D_c$ = $50\%$ que al oscilar tan extremadamente rápido entre esas dos condiciones nos generaría, a modo práctico una onda plana de un $50\%$ del voltaje total, o sea $2.5V$ (Onda roja) llamada voltaje efectivo.\\
Con el $D_c$ podemos generar un $V_e$ desde $0V$ hasta los $5V$ ganamos la capacidad de generar ondas escalonadas, cuando modifiquemos el $D_c$ modificamos el $V_e$. Ahora tenemos que aprender como modificar el $D_c$ para poder generar algo parecido a un seno.


\subsection*{Tabla de consulta}
El seno es una función periódica, con periodo T = $\frac{1}{F} = 2\pi$. Si calculamos $sen(2)$ es exactamente lo mismo que calcular $sen(2+2\pi)$, con esto en mente podríamos no precisar de calcular cada vez el seno de un numero para generar nuestra onda sino tener almacenado el resultado de un $f(x)=y$ como la tabla siguiente que el valor de $x$ comienza en $0$° y va escalando de a $22.5$° (Este escalado es conocido como $\Delta x$) hasta llegar a los $337.5$° para luego reiniciar el ciclo y volver a leer desde $0$°.\\
Si llenamos una tabla con estos valores podemos conseguir una tabla de consulta, a la cual llegaríamos y preguntaríamos: \\
Señora tabla ¿Cual es el resultado de $sen(22.5$°$)$? y ella nos responderá extremadamente rápido ¡$0.7071$!
\begin{table}[h!]
	\begin{tabular}{|c|c|c|c|} 
		\hline
		 0.0000 & 0.3827 & 0.7071 & 0.9239 \\ \hline
	     1.0000 & 0.9239 & 0.7071 & 0.3827 \\ \hline
		 0.0000 &-0.3827 &-0.7071 &-0.9239 \\ \hline
		-1.0000 &-0.9239 &-0.7071 &-0.3827  \\\hline
	\end{tabular}
\end{table}\\
Esta tabla de senos con $\Delta x = 22.5$° genera la o\\
\begin{tikzpicture}
	\draw (-1.5,-1.5) -- (10.9,-1.5) -- (10.9,3.1) -- (-1.5,3.1) -- (-1.5,-1.5);
	\begin{axis}[
		legend style={nodes = {scale=0.6, transform shape}, at={(1.11,-0.2)}},
		legend entries={$Seno\ Sobre\ Tabla\ De\ Consulta$\\$Seno\ Sobre\ Tabla\ De\ Consulta\ 36\ Elementos$\\},
		ymajorgrids=true,
		width=11cm,
		height=4cm,
		x axis line style={-stealth},
		y axis line style={-stealth},
		title={Senos sobre tabla de consulta},
		ymax = 1.5,xmax=18,
		ymin = -1.5, xmin = 0,
		axis lines*=center,
		xtick distance=3,
		ytick={-1,0,1},
		ylabel={Amplitud},
		xlabel near ticks,
		ylabel near ticks]
		\addplot+[blue, thick, no marks, const plot]
		coordinates{
			( 0 , 0.0000 )( 1 , 0.3827 )( 2 , 0.7071 )( 3 , 0.9239 )( 4 , 1.0000 )( 5 , 0.9239 )( 6 , 0.7071 )( 7 , 0.3827 )( 8 , 0.0000 )( 9 , -0.3827 )( 10 , -0.7071 )( 11 , -0.9239 )( 12 , -1.0000 )( 13 , -0.9239 )( 14 , -0.7071 )( 15 , -0.3827)(16, 0)
		};
		\addplot+[red, thick, no marks, const plot]
		coordinates{(0.0 , 0.0000 )( 0.5 , 0.1736 )( 1.0 , 0.3420 )( 1.5 , 0.5000 )( 2.0 , 0.6428 )( 2.5 , 0.7660 )( 3.0 , 0.8660 )( 3.5 , 0.9397 )( 4.0 , 0.9848 )( 4.5 , 1.0000 )( 5.0 , 0.9848 )( 5.5 , 0.9397 )( 6.0 , 0.8660 )( 6.5 , 0.7660 )( 7.0 , 0.6428 )( 7.5 , 0.5000 )( 8.0 , 0.3420 )( 8.5 , 0.1736 )( 9.0 , 0.0000 )( 9.5 , -0.1736 )( 10.0 , -0.3420 )( 10.5 , -0.5000 )( 11.0 , -0.6428 )( 11.5 , -0.7660 )( 12.0 , -0.8660 )( 12.5 , -0.9397 )( 13.0 , -0.9848 )( 13.5 , -1.0000 )( 14.0 , -0.9848 )( 14.5 , -0.9397 )( 15.0 , -0.8660 )( 15.5 , -0.7660 )( 16.0 , -0.6428 )( 16.5 , -0.5000 )( 17.0 , -0.3420 )( 17.5 , -0.1736 )(18, 0.0000)};
	\end{axis}
\end{tikzpicture}\\
Desde la gráfica de estas dos funciones podemos deducir que mientras más elementos tenga nuestras tablas de consulta mayor va a ser la precisión del seno.
Sabiendo el valor de los senos en un punto X podemos llegar a un valor de $D_{c}$ para poder generar una onda cuasi analógica.

\section*{Librería PWM}
	
	Ya es momento de ensuciarse las manos, vamos a hablar de código! Gracias a esta librería podemos hacer realidad este proyecto de generar ondas sinusoidales. Con lo aprendido hasta aquí y con la lectura de ``\href{https://datasheets.raspberrypi.com/pico/raspberry-pi-pico-c-sdk.pdf}{\underline{Raspberry Pico SDK}}" podemos empezar a codear, vamos a empezar a explicar nuestro código.\\
	\begin{minted}[bgcolor=black, autogobble]{c}
		#include "pico/stdlib.h"
		#include "hardware/pwm.h"
		#include "tabla.h"       //Nuestras Dos Tablas de consulta.
		#define SONIDO_OUT 0     //Pin de salida.
		#define ON 1
		#define OFF 0
		#define RESOLUCION 512   //Cantidad De Valores de Nuestra 
		//Tabla de Consulta de Senos
		#define A PWM_CHAN_A     //Abreviaturas
		#define PWM GPIO_FUNC_PWM//Abreviaturas
		
		float Hz(int frecuencia)
		{return hertz[frecuencia];}
		
		void setDuty(int valor, int canal)
		{pwm_set_chan_level(canal, A, valor);}//Mandamos nuestra señal
		//por el canal hasta el 
		//valor dado.
		
		
		
		int main() 
		{                                      
			int iterar = 0;//Creamos un canal PWM en el pin de Sonido
			int canal = pwm_gpio_to_slice_num(SONIDO_OUT);
			//Convertimos el pin de sonido de GPIO a PWM.
			gpio_set_function(SONIDO_OUT, PWM);
			pwm_set_enabled(canal, true);//Iniciamos PWM en el canal      
			while(1) //Loop Infinito
			{
				//Ponemos la ventana igual a nuestra Resolucion.
				pwm_set_wrap(canal, RESOLUCION*2);
				setDuty(valores[iterar], canal);
				//dormimos el tiempo suficiente para nuestra F.
				sleep_us(Hz(X));                
				(iterar == RESOLUCION-1) ? iterar = 0 : iterar++;
			}
		}
	\end{minted}
Con estas sencillas lineas de código podemos generar nuestras ondas, allí podemos ver dos cosas interesantes, la función $Hz$ y la $X$ dentro de ella, esa pequeña $X$ puede tomar el valor de cualquier numero entre $1$ y $20000$ para generar una onda sinusoidal de amplitud $1$ y frecuencia $X$. Ahora ¿Cómo llegamos a la función Hz? Esta es una simple tabla de consulta, nuevamente. 
\section*{¡Tablas de Consulta al ataque!}
	Como hemos observado las tablas de consulta son una forma liviana de conocer un resultado estático sin la necesidad de calcularlo. Por ese motivo llegamos a la conclusión que la mejor forma de manejar frecuencia de nuestra onda es con otra tabla de consulta.
\section*{Controlar frecuencia de la Onda}
	Con el fin de controlar la frecuencia de nuestra onda vamos a utilizar la librería ``\href{https://raspberrypi.github.io/pico-sdk-doxygen/group__hardware__pwm.html}{\textit{hardware/pwm.h}}''.\\
	Vamos a comentarle la idea matemática tras esto.
	Para jugar con la frecuencia de una onda contamos con 3 valores;
	\begin{description}
		\item [Wrap] Es la cantidad de ciclos de reloj pasan hasta que el reloj vuelve a 0.
		\item [Step] Cuantas celdas recorremos por paso.
		\item [Sleep] Es la cantidad de tiempo que espera el procesador antes de ejecutar la próxima orden.
	\end{description}
	Vamos a comenzar a calcular:\\ 
	Nuestra tabla de senos tiene $512$ valores que van de $0$ a $1024$ por conveniencia y para limitar el uso de flotantes al menos posible.
	Tenemos un reloj $PWM$ de $125MHz$ lo que nos muestra una cantidad de $125000000$ ciclos por segundo, aproximadamente $8 Nano segundos$ por ciclo. Nuestro Wrap va a estar capado a $1024$ ciclos por conveniencia, cada vez que cambiemos el $D_{c}$ tenemos un tiempo de $Wrap*8ns$ donde no es posible hacer nada. Eso nos da una $8196ns$ donde no pasa nada.
	La fórmula para la frecuencia de nuestra onda sinusoidal es:
	\[ f =  \]
\end{document}
